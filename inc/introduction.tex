Технологии машинного обучения все глубже проникают в жизнь человека, находя свое применение в широком спектре областей человеческой деятельности~---~от детектирования редких смертельных заболеваний~\cite{rock_et_al_2022} до прогнозирования индексов климатических мод Земли~\cite{zane_et_al_2022}. Неотъемлемой составляющей каждой технологии машинного обучения является используемая модель машинного обучения~---~много-параметрическое преобразование, производящее трансформацию входных данных. В последние десятилетия наиболее широкое распространение получили искусственные нейронные сети (НС). Главным преимуществом данной модели машинного обучения является большое количество свободных параметров, при правильном выборе которых удаётся со сколь угодно высокой точностью аппроксимировать любую функциональную зависимость \cite{hornik_et_al_1989}. Стоит заметить, что из-за большого количества параметров, процесс их подбора долгое время оставался крайне тяжелой вычислительной задачей, успешно начать решать которую позволило развитие технологий многомерной оптимизации.

Ввиду того, что НС успешно решают многомерные задачи (обработка натурального языка, генерация изображений и т.д.), естественным представляется использовать НС при решении многомерных задач математической физики. В таком приложении зачастую используются полносвязные НС прямого распространения, которые получили название physics-informed neural networks \cite{raissi_et_al_2017}, хотя их отличие от остальных типов НС заключается лишь в форме минимизируемого функционала~---~в него добавлены имеющие физическую интерпретацию члены. Форма минимизируемого функционала (функционала потерь или целевого функционала) зависит от специфики решаемой физической задачи, так, если ставится задача получить приближенное решение некоторого уравнения и НС используется в качестве пробной функции, то в функционал потерь разумно включить норму невязки НС по уравнению, граничное условие, если таковое имеется, а также все прочие соотношения, которым должно удовлетворять решение.

В 90-ые годы прошлого века было предложено использовать НС для решения задач на собственные значения и собственные функции~---~в работе \cite{lagaris_et_al_1997} предлагается решать стационарное уравнение Шредингера с помощью НС с одним скрытым слоем, включая в функционал потерь квадратичную норму невязки пробной функции по уравнению. Такой подход позволил находить основное состояние, а также несколько возбужденных состояний стационарных квантово-механических систем. Подход, применяемый в \cite{lagaris_et_al_1997}, подразумевает, что каждому состоянию отвечает отдельная НС, а состояния находятся последовательно, начиная с основного (при отыскании возбужденных состояний в функционал потерь включаются члены, отвечающие за ортогональность искомого состояния к уже найденным состояниям).

В работе \cite{li_et_al_2021} предложено использовать одну НС для отыскания сразу нескольких состояний, для этого в функционал потерь добавляют члены, отвечающие за взаимную ортогональность пробных функций, отвечающим различным состояниям. Кроме того, в данной работе предлагается использовать не равномерное распределения точек в координатном пространстве, а распределение, характеризующиеся плотнотностью вероятности, пропорциональной средней по рассматриваемым состояниям плотности вероятности. Предложенный подход выглядит привлекательно с той точки зрения ускорения решения спектральных задач высокой размерности. 

Однако, работа \cite{li_et_al_2021} полна спорных моментов, главным из которых является используемый функционал потерь --- в работе \cite{erdem_2022} приводится строгое доказательство того, что используемый в \cite{li_et_al_2021} функционал потерь имеет минимумы на функциях, не являющихся решениями задачи. В работе \cite{erdem_2022} показывается, что добавление в функционал потерь среднеквадратической невязки по уравнению Шрёдингера исправляет ситуацию и позволяет получать волновые функции стационарных систем сразу для нескольких состояний. Были рассмотрены задачи низкой размерности при этом использовалось равномерное распределение точек в координатном пространстве. 

В качестве целей выпускной квалификационной работы было избрано распространение успехов работы \cite{erdem_2022} на задачи высокой размерности. В настоящей работе будут рассмотрены задачи размерности от 3 до 6, при этом будет проверена эффективность предлагаемого в \cite{li_et_al_2021} метода сэмплирования точек в координатном пространстве...
% доработать

% Кроме того, в работе \cite{erdem_2022} произведена модернизация функционала потерь (в него добавлен член, отвечающий за невязку пробной функции по уравнению), чтобы тот достигал минимумы лишь на решении задачи, и с использованием модернизированного функционала потерь успешно решены спектральные квантово-механические задачи в размерностях 1 и 2. Целью настоящей работы является распространение успехов работы \cite{erdem_2022} на более высокие размерности.